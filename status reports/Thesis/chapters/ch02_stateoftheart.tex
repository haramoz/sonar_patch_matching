%!TEX root = ../report.tex

\chapter{State of the Art}
\label{chap:state}
%\section{Deep neural network}
%From Wikipedia \cite{wikidnn} deep neural networks (DNN) are a type of artificial neural networks which can have many layers between their input and output. The DNN can be used to estimate the underlying mathematical manipulation
%which maps the output to the input in, a manner which is more or less consistent across all the samples. The network calculate for all the layers the probability of each output and then the correct label can be displayed as output 
%if the probability is above a certain threshold. Each mathematical manipulation can be considered a layer, a DNN can have many such operating layers, that is why it is named deep neural network. For example, in our scenario two sonar 
%image patches are given as input to the DNN, it will calculate different features from both image patches and at the end, it will decide if the derived features are similar then the input patches are similar.

\section{Convolutional Neural Network}
Before starting with the related works, a little introduction to the Convolutional neural network (CNN) is provided here. CNN \cite{wikicnn} is a special type of Artificial Neural Network (ANN) with Multi-Layer Perceptrons (MLP),
consisting of learn-able weights and biases. CNNs are usually trained with backpropagation algorithm in a 
supervised manner. While a CNN also expresses a single differentiable score function similar to a normal ANN, CNNs are different in architecture though, specially designed for recognizing visual patterns directly from the raw image pixels with minimal preprocessing \cite{lecun2015lenet}.
In fully-connected architecture, each node is connected all the nodes in next layer in feed-forward fashion. But inn CNN, neurons in a layer will only be connected to a small region of the layer before \cite{cs231n}.
The CNNs, based on the explicit assumption that inputs are images, are efficiently encoded to have much lesser parameters. The main building blocks of CNNs are Convolutional Layer (Conv), 
Pooling Layer, Fully-Connected Layer (FC). Now a typical example of a CNN can be denoted by \code{[Input - Conv - ReLU - MP - FC]}, where \code{Input} holds raw pixel values of the input image, \code{Conv} layer applies 
convolution operation to the local region of input and produce output; this operation is inspired by the natural response of a neuron in biological visual cortex getting stimulated. \code{ReLU} layer provides activation to each elements 
and applies a threshold (\code{max(0,x)}) so that the negative weights become zero. \code{Pooling} layer which basically down-samples the input along the spatial dimensions of supplied height and width, by mapping a cluster of neurons
from one layer to a single output in the next layer. \code{MP} or max-Pooling layer takes the maximum value from the cluster of the neurons and maps in the next layer. \code{Average} pooling takes average values of the cluster and maps to the next layer. Pooling can also be local or global. Lastly, the \code{FC} layer just connects every neuron of one layer to the next layer. Fully-Connected layers serve as the decision network and determine the overall output size.

Inspired by \cite{stateoftheart} following notations for CNN layers are used to describe components of the architectures: \code{Conv(Nf,Fw x Fh)} is a convolutional layer with Nf filters of width Fw and height Fh.
A max-pooling layer is represented by \code{MP(Pw, Ph)} where \code{Pw x Ph} is the sub-sampling size of the layer, width and height respectively and FC(n) is a fully connected layer where n represents output size.\\

In the following section, the Siamese network is briefly discussed because it is important to have an understanding of it before some of the states of the art network architectures can be explained in details. 

\section{Siamese Network}
In 1993 the concept of a new artificial neural network called Siamese network was introduced by Bromley et al. \cite{bromley1994signature}. Siamese network consists of two identical sub-networks which are joined at the output.
During training stage one of the input pairs is connected to each of the subnetworks. The main idea here is that the subnetworks (also called branches) are trained simultaneously and extract features from the inputs.
While the shared neurons are capable of measuring the distance between the two extracted feature vectors by each branch. If the predicted distance between two feature vectors is lesser than a threshold then it can be considered that the inputs are similar, otherwise non-similar. Authors used this concept to compare two signatures in \cite{bromley1994signature}. One of the signatures was previously obtained from the authentic owner (up to 6 signatures were recorded). 
This was then used to compare with a new signature to verify if both persons are same or not, as a precaution against forgery. 
Authors implemented the Siamese network to be able to extract and compare different features of the two signatures and if the output is within a threshold then
they were considered matching. If not then it was most likely a forgery. 

\section{Sonar image matching techniques}

\begin{figure}[ht]
\centering
\includegraphics[width=10cm,height=5cm]{images/densenet/siamese/sonar_diagram}
\caption{Figure is taken from S. Emberton et al. \cite{emberton2018underwater}. Light gets absorbed and scattered by the particles and objects in the water, which causes poor contrast and spectral distortion in sonar images.}
\label{fig:sonar_diagram}
\end{figure}

Sonar image patch matching is more difficult than normal optical patch matching problem. This is because sonar images have additional challenges such as non-uniform insonification, low signal-to-noise ratio, poor contrast \cite{emberton2018underwater}, low resolution,
low feature repeatability \cite{hurtos2013automatic} etc. But sonar image matching has important applications like in sonar registration, mosaicing \cite{kim2005mosaicing}, \cite{hurtos2012fourier} and mapping of seabed surface \cite{negahdaripour2011dynamic} etc. 
While Kim et al. \cite{kim2005mosaicing} used Harris corner detection and matched key-points to register sonar images, Hurtos et al. \cite{hurtos2012fourier} incorporated Fourier-based features for registration of FLS images. S. Negahdaripour
et al. \cite{negahdaripour2011dynamic} estimated mathematical models from the dynamics of object movements and it's shadows. Vandrish et al. \cite{vandrish2011side} used SIFT \cite{lowe2004distinctive} for sidescan sonar image registration.
Even though these approaches did achieve considerable success in respective goals, were found to be most effective when the rotation/translation between the frames of sonar images are comparatively smaller. Block-matching was performed on segmented sonar images by Pham et al. \cite{pham2013guided}, using Self-Organizing Map for the registration and mosaicing task. Recently, \cite{zbontar2016stereo} for stereo matching, \cite{kim2016convolutional}
for fast under-water object detection and localization, \cite{valdenegro2016objectness} objectness scoring, CNNs are being more and more used for sonar image processing. The main reason behind such a rise of CNN usage is that the CNNs can learn sonar-specific information from the sonar data directly. No complex manual feature design or rigorous data preprocessing steps are needed, which makes the task less complex but prediction accuracy achieved is actually higher.


\section{Learning similarity function}

\begin{figure}[htb]
\centering
\includegraphics[height=6cm]{images/densenet/two_channel_only_siamese.png}
\caption{Two channel network (left) and Siamese (right) CNN architectures used in Valdenegro-Toro, though the figure and inspiration is from Zagoruyko et al.\cite{zagoruyko2015learning}}
\label{fig:two_channel_only_siamese}
\end{figure}

Zagoruyko et al. have demonstrated that CNNs can be directly deployed to learn the underlying similarity function to be able to determine that two input images/patches are different instances of the same object or not, 
without any help from hand designed features. The scarcity of accurate hand designed features for sonar images motivated Valdenegro-Toro to evaluate similar approach as Zagoruyko et al. and encode a CNN for sonar image comparison.
Valdenegro-Toro \cite{stateoftheart} evaluated two architectures, a two-channel network, and a Siamese network. The architectures figure \ref{fig:two_channel_only_siamese} were based on the work from Zagoruyko et al. \cite{zagoruyko2015learning}.
A grid search was used over a predefined set of hyperparameters to encode the best performing network. The final two-channel network structure presented in the paper for predicting binary classification score was as follows,
\code{Conv(16, 5 x 5)-MP(2, 2)-Conv(32, 5 x 5)-MP(2, 2)-Conv(32, 5 x 5)-MP(2, 2)- Conv(16, 5 x 5)-MP(2, 2)-\\FC(64)-FC(32)-FC(1)}. Similarly, grid search was also conducted for Siamese network structure for classification score. 
The structure for each 
of the branches or sub-network is as follows, \code{Conv(16, 5 x 5)-MP(2, 2)-Conv(32, 5 x 5)- MP(2, 2)-Conv(64, 5 x 5)-MP(2, 2)-Conv(32, 5 x 5)-MP(2, 2)-FC(96)-FC(96)}.\\ The output features from the branches were then concatenated to form 192 
element vector. This was then passed through a decision network with \code{FC(64)-FC(1)}. For both two channel and Siamese score prediction network, the final \code{FC} layer had Sigmoid \cite{kerassigmoid} activation function and 
binary cross entropy loss \cite{kerascrossentropy} function. 

As noted from \cite{stateoftheart} not very complex network structures were used, but the obtained results are very good. %compare with older methods and findings
In theory, more advanced networks or loss functions which help to extract more discriminative and patch invariant features should improve the results even further. 