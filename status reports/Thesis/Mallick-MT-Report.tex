%
% LaTeX2e Style for MAS R&D and master thesis reports
% Author: Argentina Ortega Sainz, Hochschule Bonn-Rhein-Sieg, Germany
% Please feel free to send issues, suggestions or pull requests to:
% https://github.com/mas-group/project-report
% Based on the template created by Ronni Hartanto in 2003
%

\documentclass[thesis]{mas_report}
% \documentclass[rnd]{mas_report}
\usepackage{url}
\usepackage{graphicx}
\usepackage{subcaption}
\usepackage{wrapfig}
\usepackage{multirow}
\usepackage{color, colortbl}
\usepackage{listings}
\usepackage{hyperref}
\pdfsuppresswarningpagegroup=1
%suppresses the page group related warnings for pdfs 
\definecolor{dkgreen}{rgb}{0,0.6,0}
\definecolor{gray}{rgb}{0.5,0.5,0.5}
\definecolor{mauve}{rgb}{0.58,0,0.82}
\definecolor{lightgrey}{gray}{0.85}

\lstset{frame=tb,
  language=Python,
  aboveskip=3mm,
  belowskip=3mm,
  showstringspaces=false,
  columns=flexible,
  basicstyle={\small\ttfamily},
  numbers=none,
  numberstyle=\tiny\color{gray},
  keywordstyle=\color{blue},
  commentstyle=\color{dkgreen},
  stringstyle=\color{mauve},
  breaklines=true,
  breakatwhitespace=true,
  tabsize=3
}
\usepackage{lineno}
\linenumbers
%\usepackage{quotchap}
% ****************************************************
% THIS INFORMATION SHOULD BE UPDATED FOR YOUR REPORT
% ****************************************************
\author{Arka Mallick}
\title{Sonar Patch Matching via Deep Learning}
\supervisors{%
Prof. Dr. Paul G. Pl\"oger\\
Prof. Dr. Gerhard K. Kraetzschmar\\
Dr. Matias Valdenegro-Toro
}
\date{January 2019}
\newcommand{\code}[1]{\texttt{#1}}


% \thirdpartylogo{path/to/your/image}
%s\hypersetup{pageanchor=false}
\begin{document}
\begin{titlepage}
    \maketitle
\end{titlepage}
%\hypersetup{pageanchor=true}
%----------------------------------------------------------------------------------------
%	PREFACE
%----------------------------------------------------------------------------------------

\pagestyle{plain}


\cleardoublepage
\statementpage

\begin{abstract}
Application of underwater robots are on the rise, most of them are dependent on sonar for underwater vision. Though sonar is more effective than optical vision in difficult underwater scenarios, acoustic images
have a low signal-to-noise ratio, low resolution and therefore harder to model accurately. Recently in Valdenegro-Toro \cite{stateoftheart}, instead of modeling features manually, a Convolutional Neural Network (CNN) 
is encoded to learn general similarity function and predict two input sonar images are similar or not. This work presented much-improved prediction accuracy for sonar image matching functionality, which is instrumental in aquatic applications
like localization, mapping, even in some object detection/recognition. 

With the objective of improving the sonar image matching problem further, three state-of-the-art CNN architectures are evaluated
on the dataset from \cite{stateoftheart}, containing about 39K pairs of training data, and another 8K for testing. To ensure a fair evaluation of each network, thorough hyperparameter optimization is executed. 
Using DenseNet Two-Channel network average prediction accuracy obtained is 0.955 area under ROC curve (AUC). VGG-Siamese (with Contrastive loss function) and DenseNet-Siamese perform the prediction with an average 
AUC of 0.949 and 0.921 respectively. All these results are an improvement over the result (0.894 AUC) from Valdenegro-Toro \cite{stateoftheart}. By encoding an ensemble of DenseNet two-channel and DenseNet-Siamese 
models with the highest AUC scores, overall highest prediction accuracy obtained is 0.978 AUC.
\end{abstract}

\cleardoublepage


\begin{dedication}
 %I dedicate this thesis to my maternal uncle Subhas Chandra Ghosh. Unfortunately he is no longer with us. Without his help I would not be able to pursue my dream of higher studies.
 Dedicated to the memory of my maternal uncle,\\ Subhas Chandra Ghosh,\\
 without whose support my dream of pursuing higher\\ studies would not have been possible.\\
 May he rest in peace.
\end{dedication}


\begin{acknowledgements}
 My deepest thanks go to my supervisor Dr. Matias Valdenegro-Toro. For several reasons. From the technical aspect, his guidance and suggestions were invaluable and always spot on.
 In the beginning, I spent several months training for deep learning and exploring other prerequisites for this work. He has been very supportive and motivating throughout the period of this thesis and even much before.
 In spite of a very busy schedule, he spent a lot of hours helping me, many times even during weekends. It was a great experience working with him. Also, it was a long time aspiration to work in the domain of underwater robotics. 
 A special thanks to him for giving me this opportunity.
 
 Special thanks to Prof. Dr. Paul G. Pl\"oger for making this project possible through his supervision and invaluable inputs, also for the neural networks lesson, which was pivotal for deep learning training.
 I sincerely thank Prof. Dr. Gerhard K. Kraetzschmar for kindly agreeing to supervise this project. Along with all the technical lessons, 
 I also thank him for the wonderful opportunity of being able to take part in RoboCup competitions, which were a great experience for me to learn hands-on programming and be in touch with great minds. In the context, 
 my sincere thanks to Santosh, Oscar, and Shehzad who were great mentors and taught me many important lessons. I also thank Prof. Dr. Rudolf Berrendorf and the entire cluster team for providing such great computational environment for all the students. Without cluster, I could have never finished this work.
 
 I must thank my best friends Swarit, Pranika, Meemansa, Aaqib, Nour, and Arshia for making my stay in Bonn stress-free and enjoyable. I also thank Adhideb and Abhilash, without their help, I would not dare to dream about masters in Germany.
 Finally, I thank my family and my partner Arshia for being supportive and encouraging throughout and for being there always. 
  
\end{acknowledgements}


\tableofcontents
\listoffigures
\listoftables

%-------------------------------------------------------------------------------
%	CONTENT CHAPTERS
%-------------------------------------------------------------------------------

\mainmatter % Begin numeric (1,2,3...) page numbering

\pagestyle{mainmatter}

\subfile{chapters/ch01_introduction}
\subfile{chapters/ch02_stateoftheart}
\subfile{chapters/ch03_methodology}
\subfile{chapters/ch05_evaluation}
\subfile{chapters/ch06_results}
\subfile{chapters/ch07_conclusion}


%-------------------------------------------------------------------------------
%	APPENDIX
%-------------------------------------------------------------------------------

\begin{appendices}
\subfile{chapters/appendix}
\end{appendices}

\backmatter

%-------------------------------------------------------------------------------
%	BIBLIOGRAPHY
%-------------------------------------------------------------------------------
\addcontentsline{toc}{chapter}{References}
\bibliographystyle{plainnat} % Use the plainnat bibliography style
\bibliography{bibliography.bib} % Use the bibliography.bib file as the source of references

\end{document}
